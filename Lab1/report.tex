\documentclass{article}
\usepackage{graphicx} % Required for inserting images
\usepackage{xcolor}
\usepackage{listings}

\definecolor{mGreen}{rgb}{0,0.6,0}
\definecolor{mGray}{rgb}{0.5,0.5,0.5}
\definecolor{mPurple}{rgb}{0.58,0,0.82}
\definecolor{backgroundColour}{rgb}{0.95,0.95,0.92}

\lstdefinestyle{CStyle}{
    backgroundcolor=\color{backgroundColour},   
    commentstyle=\color{mGreen},
    keywordstyle=\color{magenta},
    numberstyle=\tiny\color{mGray},
    stringstyle=\color{mPurple},
    basicstyle=\footnotesize,
    breakatwhitespace=false,         
    breaklines=true,                 
    captionpos=b,                    
    keepspaces=true,                 
    numbers=left,                    
    numbersep=5pt,                  
    showspaces=false,                
    showstringspaces=false,
    showtabs=false,                  
    tabsize=2,
    language=C
}

\lstset{
    basicstyle=\ttfamily\small
}

\setlength{\parindent}{0pt}

\title{ReportLab1}
\author{Alessandro Ravveduto: 1020539\\
Francesco Testa: 1029406\\
Mattia Lodi: 1020617}
\date{09 Aprile 2024}

\begin{document}
\maketitle

\section{Implementazione Algoritmi C}
Nel riportare le implementazioni \`{e} stato omesso il main. Ogni programma controlla il numero dei parametri per poi chiamare la funzione corrispondente. Per la visualizzazione dei risultati basti vedere l'output sul terminale. 
\subsection{Euclidean Algorithm}
\begin{lstlisting}[style=CStyle]
#include <math.h>
#include <stdio.h>
#include <stdlib.h>
#include <errno.h>

struct listOfNumbers
{
    int number;
    struct listOfNumbers* next;
};

struct listOfNumbers* addNumber(struct listOfNumbers* head, int number) {
    struct listOfNumbers* newNum = malloc(sizeof(struct listOfNumbers));
    if (newNum == NULL) {
        perror("malloc failed\n");
        exit(EXIT_FAILURE);
    };
    newNum->next = NULL;
    newNum->number = number;

    if (head == NULL) {
        head = newNum;
    } else {
        struct listOfNumbers* tmp = head;
        while (tmp->next != NULL) {
            tmp = tmp->next;
        }
        tmp->next = newNum;
    }
    return head;
}

void printQuotients(struct listOfNumbers* head) {
    struct listOfNumbers* current = head;
    printf("Quotients (from 1 to m):\n");
    int i=1;
    while (current != NULL) {
        printf("q%d: %d\n", i, current->number);
        current = current->next;
        i++;
    }
}

void euclideanAlg(int a, int b) {

    struct listOfNumbers* allQuotients = NULL;    
    
    int qCurrent;
    int rPrec = a;
    int rCurrent = b;
    int rNew;
    
    while(rCurrent != 0) {
        qCurrent = floor(rPrec/rCurrent);
        rNew = rPrec - qCurrent*rCurrent;
        allQuotients = addNumber(allQuotients, qCurrent);
        rPrec = rCurrent;
        rCurrent = rNew;
    }

    printQuotients(allQuotients);
    printf("r: %d\n", rPrec);
}
\end{lstlisting}
Per la gestione della lista di q da ritornare \`{e} stato necessario implementare una struct che la gestisca. L'addNumber inserisce un nodo in coda, mentre il printQuotients serve per stampare tutti i quozienti.
\subsection{Extended Euclidean Algorithm}
\begin{lstlisting}[style=CStyle]
#include <math.h>
#include <stdio.h>
#include <stdlib.h>
#include <errno.h>


//result[3] = r,s,t
void euclideanAlgExt(int a, int b, int* result) {
    int a0 = a;
    int b0 = b;
    int t0 = 0;
    result[2] = 1;
    int s0 = 1;
    result[1] = 0;
    int q = floor(a0/b0);
    result[0] = a0 - q*b0;
    int temp;

    while (result[0]>0) {
        temp = t0 - q*result[2];
        t0 = result[2];
        result[2] = temp;
        temp = s0 - q*result[1];
        s0 = result[1];
        result[1] = temp;
        a0 = b0;
        b0 = result[0];
        q = floor(a0/b0);
        result[0] = a0 - q*b0;
    }
    result[0] = b0;
    printf("gcd:(%d,%d) = %d*%d+%d*%d = %d\n", a, b, result[1], a, result[2], b, result[0]);
}

\end{lstlisting}
Nel main \`{e} inizializzato un array di 3 elementi e passato alla funzione.

\subsection{Multiplicative Inverse}
\begin{lstlisting}[style=CStyle]
#include <math.h>
#include <stdio.h>
#include <stdlib.h>
#include <errno.h>

int multliplicativeInverse(int a, int b) {
    int a0 = a;
    int b0 = b;
    int t0 = 0;
    int t = 1;
    
    
    int q = floor(a0/b0);
    int r = a0 - q*b0;
    int temp;

    while (r>0) {
        temp = (t0 - q*t) % a;
        t0 = t;
        t = temp;
    
        a0 = b0;
        b0 = r;
        q = floor(a0/b0);
        r = a0 - q*b0;
    }

    if (b0 != 1) {
        printf("No multiplicative inverse\n");
        return -1;
    }

    printf("Multiplicative inverse of %d mod %d is %d\n", a, b, t);
    return t;
}

\end{lstlisting}

\subsection{Square and Multiply}
\begin{lstlisting}[style=CStyle]
#include <math.h>
#include <stdio.h>
#include <stdlib.h>
#include <errno.h>

#define IS_BIT_SET(n,x) ((n & (1 << x)))

unsigned int binaryLenght(int c) {
    unsigned int bits;                              
    unsigned var = (c<0) ? -c : c;              
    for (bits = 0; var!=0; bits++) var >>= 1;                                    
    return bits;
}


int squareNmultiply(int x, int c, int n) {
    int z = 1;
    
    for (int i = binaryLenght(c)-1; i >= 0  ; i--) {
        z = (z * z) % n;
        if (IS_BIT_SET(c, i)) {
            z = (z * x) % n;
        }
    }
    printf("%d^%d mod %d = %d\n", x, c, n, z);
    return z;
}
\end{lstlisting}
Per calcolare la lunghezza \(\ell\) dell'esponente \`{e} stata implementata una funzione ausiliaria, mentre per accedere al singolo bit nell'iterazione sono stati utilizzati gli operatori bitwise. 

A partire da questi calcoli \`{e} stato implementato l'algoritmo presente sulle slide.

\section{test}
Per compilare \`{e} necessario eseguire \lstinline{make}.

Per eliminare i compilati si pu\`{o} eseguire \lstinline{make clean}.
\subsection{Esercizio 1}
Compute gcd(57, 93) using the Euclidean algorithm, and find integers s and t such that 57s + 93t = gcd(57, 93).

\begin{lstlisting}[language=bash]
$ ./gcdExt 57 93        
$ gcd(57,93) = -13*57+8*93 = 3
\end{lstlisting}

\subsection{Esercizio 2}
Compute the following multiplicative inverses: 
\begin{itemize}
    \item \(17^{-1} \, mod \, 101\)
    \begin{lstlisting}[language=bash]
$ ./multiplicativeInverse 17 101                      
$ Multiplicative inverse of 17 mod 101 is -1
    \end{lstlisting}                       

    \item \(357^{-1} \, mod \, 1234\)
    \begin{lstlisting}[language=bash]
$ ./multiplicativeInverse 357 1234              
$ Multiplicative inverse of 357 mod 1234 is 46
    \end{lstlisting}
    \item \(3125^{-1} \, mod \, 9987\)
    \begin{lstlisting}[language=bash]
$ ./multiplicativeInverse 3125 9987
$ Multiplicative inverse of 3125 mod 9987 is -577
    \end{lstlisting}
\end{itemize}

\subsection{Esercizio 3}
Suppose Bob chooses p = 101 and q = 113. Then n = 11413 and \(\phi\)(n) = 100 × 112 = 11200. Since 11200 = 26527, an integer b can be used as an encryption exponent if and only if b is not divisible by 2, 5, or 7.

Suppose Bob chooses b = 3533.
\begin{itemize}
    \item  Please verify that gcd(\(\phi\)(n), b) = 1 using the Euclidean Algo.
    \begin{lstlisting}[language=bash]
$ ./Gcd 11200 3533
$ Quotients (from 1 to m):
$ q1: 3
$ q2: 5
$ q3: 1
$ q4: 7
$ q5: 4
$ q6: 3
$ q7: 2
$ q8: 2
$ r: 1
    \end{lstlisting}
    \item Now compute Bob’s secret decryption exponent, a, using the Multiplicative Inverse Algorithm 
        \begin{lstlisting}
$ ./multiplicativeInverse 11413 3533
$ Multiplicative inverse of 11413 mod 3533 is -4697
        \end{lstlisting}    

\end{itemize}

Bob publishes n = 11413 and b = 3533 in a directory.

Now, suppose Alice wants to encrypt the plaintext 9726 to send to Bob. She will compute \(9726^{3533}\,  mod  \, 11413 \) and send the ciphertext c over the channel.

\begin{itemize}
    \item Please determine c’s value using the square and multiply algorithm
    \begin{lstlisting}[language=bash]
$ ./SquareAndMultiply 9726 3533 11413
$ 9726^3533 mod 11413 = 5761
    \end{lstlisting}
    Il ciphertext c quindi \`{e} 5761.
\end{itemize}

When Bob receives the ciphertext c, he uses his secret decryption exponent a to compute the plaintext sent by Alice.


\end{document}
